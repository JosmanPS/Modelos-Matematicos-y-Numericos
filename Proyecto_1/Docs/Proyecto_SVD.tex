%%%%%%%%%%%%%%%%%%%%%%%%%%%%%%%%%%%%%%%%%%%%%%%%%%%%%%%%%%%%%%%%%%%%%%%%%
\documentclass[12pt]{article}
%\usepackage[spanish]{babel}
\textwidth     =  7.0in
\textheight    =  9.0in
\oddsidemargin = -0.2in
\topmargin     = -.5in

%%%%%%%%%%%%%%%%%%%%%%%%%%%%%%%%%%%%%%%%%%%%%%%%%%%%%%%%%%%%%%%%%%%%%%%%

\usepackage{amsmath, amsthm, amssymb, latexsym}
\newtheorem{lma}{Lema}
\newtheorem{thm}{Teorema}
\newtheorem{prb}{Prueba}
%----------------------------------------------------------------------
%
% Archivo LaTeX para el proyecto de reconocimiento de patrones
%
% curso:       Modelos matematicos y numericos
% instructor:  Jose Luis Morales
%              Departamento de Matematicas
%              I T A M
%              agosto 2015
%
%----------------------------------------------------------------------

\newcommand{\real}{\mathbb{R}}
\newcommand{\complex}{\mathbb{C}}
\newcommand{\feal}{\mathbb{F}}
\newcommand{\ul}{\underline}
%
\newcommand{\noi}{\noindent}
\newcommand{\pa}{\partial}
%
\newcommand{\bea}{\begin{eqnarray}}
\newcommand{\eea}{\end{eqnarray}}
%
\newcommand{\beas}{\begin{eqnarray*}}
\newcommand{\eeas}{\end{eqnarray*}}
%
\newcommand{\non}{\nonumber}
%
\newcommand{\bit}{\begin{itemize}}
\newcommand{\eit}{\end{itemize}}
%
\newcommand{\bee}{\begin{enumerate}}
\newcommand{\eee}{\end{enumerate}}
%
\newcommand{\bed}{\begin{description}}
\newcommand{\edd}{\end{description}}
%
\newcommand{\bec}{\begin{center}}
\newcommand{\edc}{\end{center}}
%
\begin{document}
\title{Modelos matem\'aticos y num\'ericos}
\author{Proyecto \#1: Reconocimiento de patrones}
\date{Instructor: JL Morales}
\maketitle


\subsubsection*{Descomposici\'on en valores singulares}

\begin{thm} [Descomposici\'on en valores singulares] \label{svd} 
 
 Si $A$ es una matriz real de $m \times n$, entonces existen matrices ortogonales
\[
   U = [u_1, \ldots, u_m] \in \real^{m \times m}, \quad
   V = [v_1, \ldots, v_n] \in \real^{n \times n} 
\]
tales que 
\[
  U^TAV = {\rm diag}(\sigma_1, \ldots, \sigma_p) \in \real^{m \times n}, \quad   p = \min \{m,n\}
\]
en donde $\sigma_1 \ge \sigma_2 \ge \ldots \ge \sigma_p \ge 0$.
\end{thm}

\begin{prb}

\bigskip

Consideremos la norma matricial inducida por la norma vectorial euclidiana; entonces es claro que existen 2 vectores unitarios 
$x \in \real^n$, $y \in \real^m$ con la propiedad 
\[
    Ax = \sigma y, \quad \sigma = ||A||_2,
\]
ahora podemos construir matrices ortogonales $U, V$ a partir de los vectores $x$, $y$ como sigue
\[
   V = \left [\, x \,|\,  V_2   \, \right ], \quad  U = \left [ \,y \,|\,  U_2  \, \right ].
\]
De la construcci\'on anterior es f\'acil probar que
\[
   U^TAV = \left [ \begin{array}{cc}  \sigma & w^T \\
                                         0   & B  
                   \end{array} 
            \right ] = A_1   
\]
en donde $w \in \real^{n-1}$ y $B \in \real^{(m-1) \times (n-1)}$;
de la desigualdad 
\[
   \left |\left | A_1   \left [ \begin{array}{c} \sigma \\ w \end{array} \right ] \right |\right|_2^2  \ge ( \sigma^2 + w^Tw )^2
\]
podemos concluir que $||A_1||_2^2 \ge  \sigma^2 + w^Tw$. Sin embargo $\sigma^2 = ||A||_2^2 = ||A_1||_2^2$, por lo tanto $w=0$. Es decir que $A_1$ tiene la estructura diagonal por bloques
\[
 A_1 = \left [ \begin{array}{cc}  \sigma &    0 \\
                                         0   & B  
                   \end{array} 
       \right ]
\]

Un argumento inductivo evidente completa la prueba del resultado. \hfill $\square$

\end{prb}

\subsubsection*{Propiedades}

La DVS revela pr\'acticamente toda la estructura algebraica de una matriz. Por ejemplo:
\begin{enumerate}
  \item Los valores singulares de $A$ son las longitudes de los semi ejes del hiperelipsoide 
        \[
           H = \left \{ Ax : ||x||_2 =1 \right \}
        \]
         definido por $A$.
  \item Si definimos $r$ como 
  \[
     \sigma_1 \ge \cdots \sigma_r > \sigma_{r+1} = \cdots = \sigma_p = 0,
  \]
  entonces 
    \begin{enumerate}
     \item $\mbox{rango}(A) = r$
     \item $K(A) = \mbox{gen} \left \{ v_{r+1}, \ldots, v_n \right \}$
     \item $R(A) = \mbox{gen} \left \{ u_1, \ldots, u_r \right \}$
    \end{enumerate}
  \item La matriz $A$ se puede representar como una suma de matrices de rango 1 (desarrollo DVS)
         \begin{eqnarray} \label{suma}
            A = \sum_{i=1}^r \sigma_i u_i v_i^T.
         \end{eqnarray}
  \item Diversas relaciones entre  $||A||_{\rm F}$ y $||A||_2$
        \begin{enumerate}
         \item $||A||_{\rm F} = \sigma_1^2 + \cdots + \sigma_p^2$, \quad $p = \min\{m,n\}$
         \item $||A||_2 = \sigma_1$
         \item $\displaystyle{ \min_{||x||_2 = 1} ||Ax||_2 = \sigma_n}$, \quad $(m \ge n)$.
        \end{enumerate}
\end{enumerate}


\subsubsection*{Ejercicios}

\begin{enumerate}
 \item Investigar c\'omo se construye la DVS en forma num\'ericamente estable. ?`Cu\'al es la complejidad del m\'etodo m\'as difundido?
 \item Considerar el problema de m\'{\i}nimos cuadrados $Ax = b$, en donde $A$ es una matriz real de $m \times n$ de rango completo por columnas ($m>n$).  Obtener la expresi\'on 
 \[
         \bar{x} = (A^TA)^{-1} A^T b
 \]
 mediante argumentos de \'algebra lineal.
 \item Interpretar el m\'etodo de an\'alisis de componentes principales en el contexto de la DVS.
 \item Investigar el significado e impacto pr\'actico de la factorizaci\'on de Schur (matrices reales y complejas).
\end{enumerate}
 

\subsubsection*{Proyecto}
  
%En el siguiente proyecto vamos a utilizar las propiedades de la DVS para resolver un problema pr\'actico relevante.

Clasificar autom\'aticamente d\'{\i}gitos manuscritos es un problema muy difundido en el \'area de reconocimiento de patrones. La aplicaci\'on m\'as popular consiste en
automatizar la lectura de los c\'odigos postales en piezas de correo. 

 En este proyecto, los d\'{\i}gitos consisten de im\'agenes de 20 $\times$ 20 pixeles en tonos de gris; para fines pr\'acticos las im\'agenes se tratan como vectores en $\real^{400}$, obtenidos apilando las 20 columnas de una imagen.  Para cada d\'{\i}gito $d$,  $d \in \{0,1, \ldots, 9\}$, tenemos una muestra de 500 elementos organizada como una matriz de $400 \times 500$.
 
%Una manera de resolver el problema anterior es mediante la DVS.
%
%\begin{quotation}
% {\em  A partir de un conjunto de d\'{\i}gitos clasificados manualmente (muestra de entrenamiento), clasificar un conjunto de d\'{\i}gitos desconocidos (conjunto de prueba).}
%\end{quotation}
%
Consideremos un d\'{\i}gito particular, digamos $d \in \{0,1, \ldots, 9\}$. Si obtenemos la DVS de $A^d$, entonces la propiedad (\ref{suma}) 
         \[
            A^d = \sum_{i=1}^r \sigma_i u_i v_i^T.
         \]
indica que cada columna de $A^d$, denotada como $a_j$,  se puede expresar como
\[
      a_j = \sum_{i=1}^m (\sigma_i v_{ij}) u_i,
\]
en donde $v_{ij}$ es la $j$-\'esima componente del vector $v_i$. La expresi\'on anterior sugiere que cada columna se puede aproximar por los primeros $K$ t\'erminos de la suma, es decir
\[
       a_j \approx \sum_{i=1}^K (\sigma_i v_{ij}) u_i.
\]
Notar que los primeros vectores singulares de $A^d$ contienen la {\em informaci\'on dominante}. 

Ahora podemos identificar un d\'{\i}gito desconocido, digamos $z$, resolviendo el siguiente problema de m\'{\i}nimos cuadrados (MC).
\[
    \mbox{minimizar}  \quad || z - U_K \alpha ||_2^2,
\]
en donde $U_K$ es la submatriz de $U$ formada por los primeros $K$ vectores singulares. La soluci\'on del problema de MC es 
\[
    \alpha = U_K^Tz,
\]
mientras que la norma del residuo es 
\begin{eqnarray}  \label{MC}
  ||r(z)||_2 = || z - U_KU_K^Tz||_2
\end{eqnarray}
 Es inmediato concluir que si $||r(z)||_2$ es muy peque\~na, entonces probablemente $z = d$.

El razonamiento anterior se puede formular como el siguiente algoritmo:
\begin{enumerate}
 \item Inicio. 
       \begin{itemize}
         \item Calcular la DVS de las matrices $A^j, \quad j=0, \ldots, 9$.
       \end{itemize}
 \item Elegir aleatoriamente $z$, un d\'{\i}gito de la muestra completa
       \begin{itemize}
         \item resolver (\ref{MC}) para cada d\'{\i}gito
         \item asignar  $z$ a la clase con menor error $||r(z)||_2$.
         \item verificar si el d\'{\i}gito fue identificado correctamente.
       \end{itemize}
 \item Repetir 2. para 100 elecciones de $z$ y obtener la tasa de \'exito
\end{enumerate}

\subsubsection*{Reporte t\'ecnico}

 Utilizar parte del archivo \LaTeX \, de este documento para escribir un reporte en el que se discutan los resultados obtenidos. La estructura sugerida para el reporte es:
 \begin{enumerate}
  \item Introducci\'on. El problema por resolver. Relevancia.
  \item Fundamentos matem\'aticos. El teorema de existencia de la DVS + la correspondiente prueba.
  \item Fundamentos num\'ericos. La realizaci\'on pr\'actica de la DVS mediante un algoritmo  num\'ericamente estable.
  \item Experimentos. Reportar la tasa de \'exito para diferentes valores de $K$. 
  \item Conclusiones.
 \end{enumerate}


\enddocument




